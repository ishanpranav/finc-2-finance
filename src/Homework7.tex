\documentclass[12pt]{article}
\usepackage[english]{babel}
\usepackage[letterpaper,top=2cm,bottom=2cm,left=3cm,right=3cm,marginparwidth=1.75cm]{geometry}
\usepackage{amsmath}
\usepackage{amsfonts}
\usepackage{graphicx}
\DeclareMathOperator{\pv}{pv}
\DeclareMathOperator{\pva}{pva}
\renewcommand{\theenumi}{\alph{enumi}}
\makeatletter
\renewcommand{\thesection}{Part \Roman{section}}
\renewcommand{\thesubsection}{\arabic{subsection}}
\makeatother
\title{FINC-UB 1 Homework 7}
\author{Ishan Pranav}
\date{November 11, 2023}
\begin{document}
\maketitle
\section{Arbitrage}
\subsection{The stock PolarBear.com}
\begin{enumerate}
\item Assume no arbitrage. Suppose the price of PolarBear.com on the North Pole Exchange is \$18. By the principle of no arbitrage, we have the law of one price. So the price of PolarBear.com on the South Pole Exchange is \$18.
\item Suppose the price of PolarBear.com on the North Pole Exchange is \$18 and the price on the South Pole Exchange is \$17. Assume no taxes, trading costs, or collateral requirements. Our initial equity for the investment is 0. On the North Pole Exchange, we enter a short position of PolarBear.com (invest $-\$18$) and receive \$18 in cash. Note our equity remains $(-\$18)+\$18=0$. Then, we invest \$17 in PolarBear.com on the South Pole Exchange. Now our equity is still $(-\$18)+\$17+\$1=0$. However, we have received a net cash flow of \$1 in the present. In the future, we repay our liability (1 share of PolarBear.com) with the share that trades on the South Pole Exchange, so the future cash flow is 0. Therefore, with no equity invested, and no risk, we obtain a positive cash flow in the present and no cash flow in the future. Hence, this is a successful arbitrage. It yields \$1 of profit.
\item Assume no arbitrage. Let the price of PolarBear.com on the North Pole Exchange be \$18 and suppose the North Pole Exchange charges a \$2 fee for buying and selling. Let $p$ represent the price on the South Pole Exchange. Then $\$16\leq p\leq\$20$. 

Assume, for the sake of contradiction, that $p<\$16$. Then the price on the North Pole Exchange is greater than that on the South Pole Exchange. The present cash flows are the proceeds from taking a short position on the North Pole Exchange (\$18), less the \$2 trading fee, less the cost of investing on the South Pole Exchange ($p$), or $(\$18-\$2-p)=(\$16-p)$. Since $p<\$16$, the present cash flow is positive. Of course, the future cash flow is 0 because, in the future, we can use our share on the South Pole Exchange to pay off our liability. A successful arbitrage trade is possible, which contradicts the principle of no arbitrage. Thus our assumption is false: Therefore $p\geq\$16$.

Assume, again for the sake of contradiction, that $p>\$20$. Then the price on the South Pole Exchange is greater than that on the North Pole Exchange. The present cash flows are $(p-\$18-\$2)=(p-\$20)$. Since $p>\$20$, the present cash flow is positive. Again, the future cash flow is 0. The existence of a successful arbitrage trade contradicts the principle of no arbitrage. Thus our assumption is false: Therefore $p\leq\$20$.

Hence, the price on the South Pole Exchange is between \$16 and \$20.
\end{enumerate}
\subsection{Rain or shine}
Let \textsc{RAIN} and \textsc{SUN} be two securities such that RAIN pays \$100 if there is rain during the World Cup final and such that SUN pays \$100 if there is no rain during the World Cup final. Let $t_0$ represent the time 1 year before the World Cup final and $t_1$ represent the time of the World Cup final. Let the price of \textsc{RAIN} at time $t_0$ be \$23 and the price of \textsc{SUN} at time $t_0$ be \$70.
\begin{enumerate}
\item At time $t_0$, we purchase \textsc{RAIN} and \textsc{SUN}, so our initial investment is $\$23+\$70=\$93$. At time $t_1$, either there is rain or there is no rain.

Suppose there is rain. Then we receive \$100 from \textsc{RAIN}.

Suppose instead there is no rain. Then we receive \$100 from \textsc{SUN}.

In all cases, we receive \$100. So, our payoff is $(\$100-\$93)=-\$7$. We expect a payoff of \$7, or a return of about 7.526\%, regardless of the weather.
\item Assume no arbitrage. Then the price of a 1-year zero coupon bond (whose issuer is unspecified) is less than or equal to 93\% of its face value. In other words, the yield of a 1-year zero-coupon bond is greater than or equal to 7.526\%.

The combined portfolio of \textsc{RAIN} and \textsc{SUN} has an expected return of about 7.526\% with no volatility. Since there is no volatility, there is no systemic risk and no idiosyncratic risk. By the principle of no arbitrage, the return of the portfolio is equal to the risk-free rate. Thus, a risk-free security has an annualized yield of about 7.526\%. A 1-year zero coupon bond whose issuer is unspecified has a volatility greater than or equal to zero, so its yield is greater than or equal to 7.526\%.

Consider a \$100, 1-year, zero-coupon U.S. Treasury bill. We expect its price to be \$93.
\item Suppose that a 1-year zero-coupon bond (assume a face value of \$100) is trading at \$90. Assume no taxes, trading costs, collateral requirements, or credit risk. At time $t_0$, we enter a short position in the portfolio of \textsc{RAIN} and \textsc{SUN}, with \textsc{RAIN} providing \$23 and \textsc{SUN} providing \$70. At the same time $t_0$, we invest these funds in the 1-year zero-coupon bond, costing \$90. So our cash flow at $t_0$ is $\$23+\$70-\$90=\$3$. At time $t_1$, we receive \$100 from the maturity of the zero-coupon bond, and there is either rain or no rain. In both cases, our liability is \$100. So our cash flow at time $t_1$ is $\$100-\$100=0$. Therefore, with no equity invested, and no risk, we obtain a positive cash flow at $t_0$ and no cash flow at $t_1$. Hence, this is a successful arbitrage. It yields \$3 of profit.
\item No. The profit from the arbitrage described in part (c) is \$3, which is less than the total trading cost of $\$2+\$2=\$4$. After fees, the trade makes a \$1 loss.
\end{enumerate}
\section{Fixed-income securities}
\subsection{A \$1000, 5-year, zero-coupon Treasury bond costs \$800}
\begin{enumerate}
\item For a zero-coupon bond, the yield to maturity $y$ is the same as the annualized holding period return $r_{\rm HPR}$, or about 4.56\%. \[y=r_{\rm HPR}=\left(\frac{\$1000}{\$800}\right)^{\frac{1}{5}}-1\approx 4.5640\dots\%.\]
\item The yield to maturity on comparable zero-coupon bonds increases to 7\% immediately. Suppose that the bond is sold after 1 year. By the principle of no arbitrage, the sale price $p_1$ is the present value of a \$1000, 4-year, zero-coupon Treasury bond with a yield to maturity of 7\%. \[p_1=\$1000\left(\frac{1}{1+7\%}\right)^4\approx\$762.90.\] So the annualized holding period return $r_{\rm HPR}$ is about $-4.64\%$. \[r_{\rm HPR}=\left(\frac{\$762.90}{\$800}\right)-1\approx-4.6381\dots\%.\]
\item Suppose instead that the bond is sold after 2 years. By the principle of no arbitrage, the sale price $p_2$ is the present value of a \$1000, 3-year, zero-coupon Treasury bond with a yield to maturity of 7\%. \[p_2=\$1000\left(\frac{1}{1+7\%}\right)^3\approx\$816.30.\] So the annualized holding period return $r_{\rm HPR}$ is about 2.04\%. \[r_{\rm HPR}=\left(\frac{\$816.30}{\$800}\right)-1\approx 2.0372\dots\%.\]
\item After 3 years the yield to maturity on comparable zero-coupon bonds decreases to 3\% immediately. Suppose the bond is sold at that time. By the principle of no arbitrage, the sale price $p_3$ is the present value of a \$1000, 2-year, zero-coupon Treasury bond with a yield to maturity of 3\%. \[p_3=\$1000\left(\frac{1}{1+3\%}\right)^2\approx\$942.60.\] So the annualized holding period return $r_{\rm HPR}$ is about $17.82\%$. \[r_{\rm HPR}=\left(\frac{\$942.60}{\$800}\right)-1\approx 17.8245\dots\%.\]
\item Suppose instead the bond is sold after 4 years. By the principle of no arbitrage, the sale price $p_4$ is the present value of a \$1000, 1-year, zero-coupon Treasury bond with a yield to maturity of 3\%. \[p_3=\$1000\left(\frac{1}{1+3\%}\right)\approx\$970.87.\] So the annualized holding period return $r_{\rm HPR}$ is about $21.36\%$. \[r_{\rm HPR}=\left(\frac{\$970.87}{\$800}\right)-1\approx 21.3592\dots\%.\]
\item Suppose instead the bond is sold after 5 years. From part (a), the annualized holding period return is about 4.56\%. \[r_{\rm HPR}=\left(\frac{\$1000}{\$800}\right)^5-1\approx 4.5640\dots\%.\]
\item The annual returns calculated in parts (b) through (f) vary but eventually equal the yield to maturity in part (a). This phenomenon is driven by the ``pull to parity.'' For zero-coupon bonds, the intermediate yields react to changes in the market interest rate based on the bond's duration and convexity. Selling the bond before maturity involves a discount or premium depending on whether the market interest rate has increased or decreased since the initial purchase. These discounts and premiums affect the annualized holding period returns calculated in parts (b) through (f). However, for zero-coupon bonds held to maturity, purchase price and face value are the only determinants of holding period return. In this case, the annualized holding period return equals the yield to maturity in part (a).
\end{enumerate}
\subsection{A \$1000, 5-year, 10\%-semiannual-coupon government bond}
\begin{enumerate}
\item Suppose the yield to maturity on similar 5-year semiannual-coupon government bonds is 8\%. Since $10\%>8\%$, this bond trades at a premium. The  price $-C_0$ is about \$1079.85. \[0=C_0+\$1000\left(\frac{1}{1+8\%}\right)^5+\$100\sum_{j=1}^5{\left(\frac{1}{1+8\%}\right)^j}.\]
\[-C_0\approx\$1079.85.\]
\item Suppose instead the yield to maturity on similar 5-year semiannual-coupon government bonds is 12\%. Since $10\%<12\%$, this bond trades at a discount. The price $-C_0$ is about \$927.90. \[0=C_0+\$1000\left(\frac{1}{1+12\%}\right)^5+\$100\sum_{j=1}^5{\left(\frac{1}{1+12\%}\right)^j}.\]
\[-C_0\approx\$927.90.\]
\item Let $-C_0=\$1030$ be the price of the bond. The yield to maturity $y$ is the same as the annualized return for the 5-year holding period.
\end{enumerate}
\end{document}